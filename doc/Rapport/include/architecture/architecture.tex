\chapter{Architecture du projet}

	\section{Arborescence du projet}

		\begin{description}
			\item[.github :]{Fichiers spécifiques à GitHub.}
			\begin{description}
				\item[workflows :]{Fichiers destinés au module d' "Actions" de GitHub. Nous nous en sommes servis pour lancer automatiquement les tests unitaires lors d'un push ou d'une pull-request.}
			\end{description}
			\item[app :]{Contient tout le code source de notre application.}
			\begin{description}
				\item[gradle:]{Wrapper de gradle.}
				\item[livre :]{Exemples de livre.}
				\item[src :]{Contient les codes sources, ressources et tests unitaires.}
				\begin{description}
					\item[main :]{Code principal de l'application.}
					\begin{description}
						\item[java :]{Code source.}
						\item[resources :]{Ressources pour l'application (images, ...).}
					\end{description}
					\item[test :]{Le code des tests unitaires.}
					\begin{description}
						\item[java :]{Code source.}
						\item[resources :]{Ressources utiles pour les tests uniquement (images, ...).}
					\end{description}
				\end{description}
				\item[build.gradle :]{Script de configuration du projet (dépendance, classe principale, ...).}
				\item[gradlew:]{Script pour les systèmes Unix afin d'exécuter le Wrapper de Gradle.}
				\item[gradlew.bat :]{Script pour les systèmes DOS afin d'exécuter le Wrapper de Gradle.}
				\item[settings.gradle :]{Configuration sur les modules à inclure, les noms de ceux-ci, etc.}
				\item[.gitattributes :]{Permet de fixer la fin de ligne pour les scripts Unix et DOS.}
			\end{description}
			\item[doc :]{Contient toute la documentations du projet, notamment le rapport.}
			\item[.gitignore :]{Fichier ignorant les changements sur certains fichiers ou dossier sur Git.}
			\item[CONVENTIONS.md :]{Conventions de nommage concernant le projet et les commits.}
			\item[LICENSE :]{Licence du projet.}
			\item[README.md :]{README pour présenter notre projet et expliquer la compilation de celui-ci.}
		\end{description}

		Pour mieux comprendre la structure de gradle les liens suivants sont utiles \url{https://guides.gradle.org/creating-new-gradle-builds/} et \url{https://docs.gradle.org/6.3/userguide/gradle_wrapper.html}

	\section{Présentation des packages}

		Notre application contenant beaucoup de classes, celles-ci sont réparties en packages que nous allons détailler :

		\begin{description}
			\item[core :]{Classes principales de l'application}
			\begin{description}
				\item[exception :]{Classes d'exceptions}
				\item[file :]{Classes utiles à la lecture, l'écriture de fichier (json et texte)}
				\begin{description}
					\item[deserializer :]{Classes qui héritent de JsonDeserializer (provient de GSON)}
					\item[json :]{Classes JSON intermédiaires pour la lecture et l'écriture avec GSON}
				\end{description}
				\item[game :]{Classes spécifiques au jeu (Personnage, Skill, BookState, ...)}
				\begin{description}
					\item[character\_creation :]{Classes qui représentent une étape de la "Création du personnage"}
					\item[player :]{Classes qui permettent de jouer au jeu (Joueur ou fourmis)}
				\end{description}
				\item[graph :]{Classes qui représentent les noeuds et les liens}
				\begin{description}
					\item[node :]{Classes pour les noeuds}
					\item[node\_link :]{Classes pour les liens}
				\end{description}
				\item[item :]{Classes qui représentent les items}
				\item[parser :]{Classes qui permettent de parser un texte pour afficher le nom de l'item ou du personnage}
				\item[requirement :]{Classes pour gérer les prérequis sur un noeud}
			\end{description}

			\item[observer :]{Classes pour le pattern observer}
			\begin{description}
				\item[book :]{Classes pour le pattern observer du livre}
				\item[fx :]{Classes pour le pattern observer des éléments JavaFx}
			\end{description}

			\item[window :]{Classes pour l'affichage avec JavaFx}
			\begin{description}
				\item[component :]{Composants réutilisables à différents endroits (dans plusieurs boites de dialogues par exemple)}
				\item[dialog :]{Les différentes boites de dialogue}
				\item[gui :]{Les différents éléments graphiques pour JavaFx (NodeFx, NodeLinkFx, PreludeFx)}
				\item[pane :]{Les différentes parties qui composent notre affichage sur la fenêtre (Partie de gauche, centrale, droite)}
			\end{description}
		\end{description}
