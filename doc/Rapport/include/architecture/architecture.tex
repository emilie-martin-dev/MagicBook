\chapter{Architecture du projet}

	\section{Arborescence du projet}



	\section{Présentation des packages}

		Notre application contenant beaucoup de classes, celles-ci sont réparties en packages que nous allons détailler :

		\begin{description}
			\item[core]{Classes principales de l'application}
			\begin{description}
				\item[exception]{Classes d'exceptions}
				\item[file]{Classes utiles à la lecture, l'écriture de fichier (json et texte)}
				\begin{description}
					\item[deserializer]{Classes qui héritent de JsonDeserializer (provient de GSON)}
					\item[json]{Classes JSON intermédiaire pour la lecture et l'écriture avec GSON}
				\end{description}
				\item[game]{Classes spécifiques au jeu (Personnage, Skill, BookState, ...)}
				\begin{description}
					\item[character\_creation]{Classes qui représentent une étape de la "Création du personnage"}
					\item[player]{Classes qui permettent de jouer au jeu (Joueur ou fourmis)}
				\end{description}
				\item[graph]{Classes qui représentent les noeuds et les liens}
				\begin{description}
					\item[node]{Classes pour les noeuds}
					\item[node\_link]{Classe pour les liens}
				\end{description}
				\item[item]{Classes qui représentent les items}
				\item[parser]{Classes qui permettent de parser un texte pour afficher le nom de l'item ou du personnage}
				\item[requirement]{Classes pour gérer les prérequis sur un noeud}
			\end{description}

			\item[observer]{Classes pour le pattern observer}
			\begin{description}
				\item[book]{Classes pour le pattern observer du livre}
				\item[fx]{Classes pour le pattern observer des éléments JavaFx}
			\end{description}

			\item[window]{Classes pour l'affichage avec JavaFx}
			\begin{description}
				\item[component]{Composants réutilisables à différents endroits (dans plusieurs boites de dialogues par exemple)}
				\item[dialog]{Les différentes boites de dialogue}
				\item[gui]{Les différents éléments graphiques pour JavaFx (NodeFx, NodeLinkFx, PreludeFx)}
				\item[pane]{Les différentes parties qui composent notre affichage sur la fenêtre (Partie de gauche, centrale, droite)}
			\end{description}
		\end{description}
