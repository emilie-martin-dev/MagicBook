\chapter{Architecture du projet}

	\section{Arborescence du projet}
      \begin{description}
    
    \item[.github :]Système d'action specifique pour github nous permettant par exemple lors d'un push/pull de tester automatiquement le projet avec des test unitaires. 
   
            
    \item[app :] Contient tous le code source de notre application.
        \begin{description}
               \item[src :] Le code des differentes classes du projet.
                
                 
                  \item[gradle:] Présence du fichier .jar pour executer gradle.
                   \item[livre :] Exemple de livre.
                    \item[settings.gradle :] Renomme la racine de notre projet.
                    \item[build.gradle :] Script de configuration qui définit notre projet et ses taches.
                    \item[graldlew.bat :]script shell conçus pour télécharger Gradle si on ne l'a pas pour windows. 
                    \item[gradlew:] script shell conçus pour télécharger Gradle si on ne l'a pas pour unix.
                  
        \end{description} 
    
	\item[doc :] Contient toutes la documentations du projet, notamment le rapport.
	
	\item[.gitignore :] Fichier ignorant certain fichier sur git afin de ne pas les commits.

	\item [CONVENTIONS.md :] Fichier texte listant les conventions auxquelles chaque membres du groupe                    c'est pliée afin de réalisé le projet.

	\item[licence :] Terme de la Licence du MIT.

    \item[README.md :] Liste les objectifs de notre projet et, explique la compilation de celui-ci.
    
    \end{description}  

	\section{Présentation des packages}
