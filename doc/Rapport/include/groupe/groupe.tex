\chapter{Travail de groupe}

	\begin{centering}
		\begin{longtable}{|p{8cm}|c|c|c|c|}
			\hline
			\rowcolor{lightgray} \centering \textbf{Tâches effectués} & \textbf{Auréline} & \textbf{Dimitri} & \textbf{Justine} & \textbf{Maxime}\\
			\hline
			\endhead
			\rowcolor{lightgray} \multicolumn{5}{|c|}{ \textbf{Lecture et enregistrement des fichiers}}\\
			\hline
			Classes pour parser le JSON& & & X & X\\
			\hline
			Lecture d'un fichier JSON & & & X & \\
			\hline
			Enregistrement d'un fichier JSON & & & X & \\
			\hline

			\rowcolor{lightgray} \multicolumn{5}{|c|}{ \textbf{Livre}}\\
			\hline
			Classe Book & & & X & \\
			\hline
			Classes pour représenter les noeuds et les liens& X & & X & \\
			\hline
			Classe BookCharacter& & & X & X\\
			\hline
			Classes pour les Requirement & X & & X & \\
			\hline
			Classes pour représenter les différents types d'items & & & X & \\
			\hline
			Classes pour la "Création du personnage" & & & X & \\
			\hline
			Classe pour représenter des skills & & & X & \\
			\hline
			Ajout des classes pour le pattern observer (uniquement celles qui concernent le livre) & & & X & X\\
			\hline

			\rowcolor{lightgray} \multicolumn{5}{|c|}{ \textbf{Jeu et export au format texte}}\\
			\hline
			Classe Jeu, partie commune au joueur et la fourmis& X & & & \\
			\hline
			Classe pour la logique de la fourmis& X & & & \\
			\hline
			Classe pour la logique du joueur & X & & & \\
			\hline
			Permettre une estimation de la difficulté du livre  & X & & & \\
			\hline
			Generation du livre en format texte & & & X & \\
			\hline
			Classe BookState & & & X & \\
			\hline
			Conception d'un Parser pour le texte des liens et des paragraphes & & & X & \\
			\hline
			Version primitive de l'estimation de la difficulté d'un livre& & & & X\\
			\hline

			\rowcolor{lightgray} \multicolumn{5}{|c|}{ \textbf{Fenêtre}}\\
			\hline
			Fenêtre principale & & X & X & \\
			\hline
			Permettre la conception d'un nouveau livre, l'ouverture d'un ancien livre sauvegarder, la sauvegarde et "sauvegarde sous" du livre courant & & & X & \\
			\hline
			Lister et permettre l'ajout d'items et de personnages sur le panel de gauche& & X & & \\
			\hline
			Permettre d'editer ou supprimer un item ou un personnage du livre & & & X & \\
			\hline
			Statistiques concernant les noeuds & & & X & \\
			\hline
			Statistique sur le niveau de difficulté du livre & X & & & \\
			\hline
			Cacher panel des statistiques si l'on décoche une case dans le menu & & & X & \\
			\hline
			Cache le panel de gauche si l'on décoche une case dans le menu& & & & X\\
			\hline
			Séparation des différentes parties de la fenêtre en plusieurs classes (LeftPane, GraphPane, RightPane) & X & & & \\
			\hline
			Composants réutilisables pour créer des personnages, une phase de la "Création d'un personnage", sélectionner une liste d'items & & & X & \\
			\hline

			\rowcolor{lightgray} \multicolumn{5}{|c|}{ \textbf{Boites de dialogues}}\\
			\hline
			Classe mère pour les boites de dialogue & X & & &\\
			\hline
			Boite de dialogue pour les noeuds & X & & & \\
			\hline
			Boite de dialogue pour les liens entre les noeuds & X & & & \\
			\hline
			Boite de dialogue pour les items & X & & & \\
			\hline
			Boite de dialogue pour les personnages & & & X & \\
			\hline
			Boite de dialogue pour le prélude & & & X & \\
			\hline
			Boite de dialogue pour la "Création du personnage" & & & X & \\
			\hline
			Boite de dialogue pour le personnage par défaut & & & X & \\
			\hline

			\rowcolor{lightgray} \multicolumn{5}{|c|}{ \textbf{Zone d'édition}}\\
			\hline
			Classe pour représenter un noeud graphique & X & & & \\
			\hline
			Ajout d'un noeud & X & & & \\
			\hline
			Modification d'un noeud & X & & & \\
			\hline
			Suppression d'un noeud & X & & & \\
			\hline
			Classe pour représenter un lien entre 2 noeuds & & & X & \\
			\hline
			Ajout d'un lien entre 2 noeuds (NodeLinkFx) & & & X & \\
			\hline
			Un lien suit les noeuds auxquelles il est attaché& & & X & \\
			\hline
			Modification d'un lien entre 2 noeuds & X & & & \\
			\hline
			Suppression d'un lien entre 2 noeuds & X & & & \\
			\hline
			Classe mère commune pour représenter un prélude et un noeud (RectangleFx) & & & X & \\
			\hline
			Permettre le déplacement des noeuds & X & & & \\
			\hline
			Détecter un clique sur un noeud ou lien (classes observer) & X & & X & \\
			\hline
			Gestion des actions en fonction du mode & X & & & \\
			\hline
			Afficher un rectangle qui représentera le prélude & & & X & \\
			\hline
			Gestion du texte de prélude & & & X & \\
			\hline
			Gestion du personnage par défaut & & & X & \\
			\hline
			Gestion de la "Conception du personnage" & & & X & \\
			\hline
			Changer le premier noeud du livre & & & X & \\
			\hline
			Répartition des différents noeuds lors de l'ouverture d'un fichier& & & X & \\
			\hline
			Gestion du niveau de zoom& & & X & \\
			\hline
			Rend le GraphPane scrollable& & & X & \\
			\hline
			Change la couleur d'un noeud en fonction de son type (normal, aléatoire, combat, victoire, ...) & X & & & \\
			\hline
			Mettre en valeur un noeud lorsque l'on passe la souris dessus & X & & & \\
			\hline

			\rowcolor{lightgray} \multicolumn{5}{|c|}{ \textbf{Autre}}\\
			\hline
			Rapport & X & & X & \\
			\hline
			Restructuration du livre fournis pour les tests (fotw.json) & & & X & \\
			\hline
			Création de tests unitaires & X & & X & \\
			\hline
			Javadoc & X & & & \\
			\hline
			Revue de code avant de merge & & & X & \\
			\hline
		\end{longtable}
	\end{centering}

	\section{Idées d'améliorations}

		Notre application n'ayant pu être terminé faute de temps, voici la liste des améliorations que nous aurions voulus faire et celles qui seraient possibles d'implémenter ensuite :

		\begin{itemize}
			\item{Concevoir deux types de fichier, l'un pour l'éditeur et l'autre pour le jeu. Le jeu serait une version épurée de celui de l'éditeur et ne contiendrait pas la position des noeuds par exemple}
			\item{Une mise à jour d'un noeud transfert correctement les différents liens (au lieu de les supprimer dans la plupart des cas)}
			\item{Vérifier que le livre est valide pour être joué}
			\item{Créer une classe mère pour les listes sur le côté gauche de l'application (Item et personnage)}
			\item{Une fois la classe mère codé, ajouter une liste à gauche pour gérer les skills dans l'éditeur}
			\item{Déclencher plus d'exception si le livre est incorrect}
			\item{Gérer les shops(jeu et gui), champs auto (jeu uniquement)}
			\item{Afficher les personnages et items inutilisés}
			\item{Indiquer si l'estimation de la difficulté est à jour ou non}
			\item{Gestion des prérequis sur les boites de dialogue des liens}
			\item{Améliorer l'intelligence de la fourmis (pouvoir estimer si un item est plus important qu'un autre, meilleure gestion des combats, ...)}
			\item{Ajouter et supprimer des skills au fil du jeu}
			\item{Ajout de paramètres aux skills (plutot que d'avoir un simple nom)}
			\item{Afficher les chemins gagnants}
			\item{Enlever ou ajouter une somme d'argent à un personnage se fait sur une monnaie précise (ex : -5 dollards, +15 euros, etc)}
			\item{"Langage" simple permettant de manier des conditions et variables pour des prérequis notamment}
			\item{Possibilité d'avoir des pnj qui pourraient nous suivre dans l'aventure pour combattre ou pour dévérouiller certains passages par exemple.}
		\end{itemize}

	\section{Bugs et problèmes connus}

		Certains et problèmes sont connus, en voici une liste non exhaustive une fois de plus :

		\begin{itemize}
			\item{Tests incomplets sur le Book et le Jeu}
			\item{Le changement d'id d'un personnage ou d'un item ne met pas à jour les différents élements du livre (noeuds, choix, ...)}
			\item{Diverses bugs visuels concernant la boite de dialogue sur le Prélude}
			\item{Le zoom ne se fait pas selon la position actuel de la souris mais du point supérieur gauche du GraphPane}
		\end{itemize}
