\chapter{Conclusion}

	Ce projet a permis aux différents membres de s'améliorer, que ce soit dans l'apprentissage d'une librairie, la mise en application des concepts vus en cours ou bien encore, dans l'art de la procrastination. Bien que nous ayons été quatre sur le projet, deux de nos camarades n'ont pas fourni suffisamment d'aide dans celui-ci. Il a fallu leur demander à de nombreuses reprises de travailler, des modifications de quelques lignes pouvaient prendre jusqu'à trois semaines pour être rendues, tout en étant parfois incomplètes. Ainsi, pour nous, le groupe était un groupe constitué de deux personnes uniquement. Le projet a été très intéressant mais nous regrettons beaucoup de ne pas avoir pu le compléter et de ne pas avoir rendu le code aussi propre que ce que nous aurions voulu. Nous sommes, par exemple, déçue de ne pas avoir pu fournir certaines fonctionnalités qui sont prises en compte dans le jeu mais pas dans l'éditeur (gestions des skills, des prérequis, ...). L'application reste cependant fonctionnelle et plutôt complète.
