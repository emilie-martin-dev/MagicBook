\chapter{Aspects techniques}

	\section{Représentation d'un livre}

		\subsection{Représentation des noeuds et des liens}

		\subsection{Quelques algorithmes}



	\section{Lecture et écriture d'un livre}


	\section{Edition graphique d'un livre}
		Afin d'éditer un livre, nous avons différents boutons permettant de changer de mode. Chaque mode permet d'ajouter, de supprimer, de sélectionner, un noeud ou un lien. Ces modes sont gérer à l'aide de la MainWindows qui est une fenêtre qui affiche tout les différents panel. Il existe trois grand panel comportant eux même d'autre petit panel. Le premier panel se nomme le "LeftPane". Il se situe à gauche et est composé des boutons de mode, d'une liste d'item existant ainsi que d'une liste de personnage existant. Le deuxième panel s'appel le "GraphPane". Il est au centre et permet d'ajouter les noeuds ainsi que les liens entre les noeuds. Il contient également le prélude. Le troisième panel se nomme le "RightPane". Il permet d'afficher les statistiques du livre comme par exemple, le nombre de noeud et l'estimation de la difficulté du livre.\\

		\subsection{MainWindows}
			La MainWindows, comme mentionné précédement, contient toute la fenêtre. Elle contient également un Menu permettant de réaliser plusieurs actions.\\
			Premièrement, dans le menu nommé "Fichier", l'utilisateur peut alors ouvrir un nouveau livre vide en cliquant sur "Nouveau". Un nouveau livre est alors ouvert en mettant tout les panel (LeftPane, GraphPane, RightPane) à jour, permettant ainsi de remmettre tout à zéro. Ce menu comporte aussi un MenuItem nommé "Ouvrir", permettant ainsi d'ouvrir un fichier Json ou txt. Si ce fichier n'est pas fonctionnel, un message d'erreur apparait. Sinon, le livre est changé en mettant tout les panel à jour. Il peut aussi enregistrer ou enregistrer-sous le livre en faisant appel au FileChooser et au File.\\
			Deuxièmement, dans le menu nommé "Livre", l'utilisateur peut "Jouer" ou "Estimer la difficulté". Ces deux MenuItem utilisent toutes les deux la classe Jeu, décrit un peu plus loin. Un autre MenuItem est aussi présent, permettant de "Générer le livre en txt". Cela permet à l'utilisateur de pouvoir avoir un livre propre contenant les paragraphes et le choix de chaque paragraphes.\\
			Troisièmement, dans le menu nommé "Affichage", l'utilisateur peut afficher ou non, le LeftPane et/ou le RightPane. Cela permet notament à l'utilisateur de mieux voir la partie "édition". Mais était surtout réalisé pour un meilleur affichage en vidéo projecteur de notre projet.\\

		\subsection{LeftPane}
			Ce panel contient tout d'abord des ToggleButton permettant de sélectionner un mode parmis cinq modes : SELECT, ADD_NODE, ADD_NODE_LINK, DELETE, FIRST_NODE. Chacun de ces mode fait appel à une méthode permettant de créer le bouton. Cette méthode prend en paramètre une image et un des modes de la classe Mode.\\
			Ensuite, un TreeItem contenant des BookCharacter est créer permettant de lister tout les personnages créé à l'aide de TreeView. Le même procédé est réalisé pour les items.\\
			Puis, pour permettre à l'utilisateur de créer des items et des personnages, un ContextMenu est ajouter au TreeView des items, puis au TreeView des personnages. Une action est implémenté pour chaque choix de ces ContextMenu: Ajout, Modification, Suppression.\\
			Si le choix d'ajout est cliqué, une boite de dialog apparait appelant CharacterDialog qui elle même appel CharacterComponent pour la création d'un personnage ou ItemDialog pour la création d'un item.\\
			Pour ce dernier, l'ItemDialog à une ChoiceBox permettant de choisir si l'item créer est une arme, un item de soin, de défense ou un simple item. Le BookItem est créer en fonction du type d'item.\\


		\subsection{GraphPane}
		Pour ajouter un noeud, on clique sur le mode "ADD_NODE",puis on clique sur le GraphPane. Une fenêtre de dialog est alors ouverte afin de renseigner le paragraphe à afficher, ainsi qu'une CheckBox permettant de sélectionner le type de noeud. L'affichage change en fonction du type de noeud sélectionner dans cette CheckBox, qui elle, est prédéfini sur "basic".\\

		Si c'est un noeud basic ou aléatoire, les renseignements suivant sont demandé: le texte à afficher, le nombre de points de vie à gagner/perdre, les items disponible, le nombre d'item disponible. Rien ne change sur la structure de la boite de dialog entre ces deux types. Mais au moment de la validation, le type de BookNode créer se fera en fonction de la valeur de la choiceBox : BoockNodeWithChoices, BookNodeWithRandomChoices. Des try/catch sont utilisé afin de changer certaine valeurs de saisie en Integer. Cela permet d'être sûr que ce sont des chiffres. Si la conversion ne peut pas se faire, une boite de dialog affichant l'erreur apparait, notifiant ainsi où l'erreur se situe.\\

		Si c'est un noeud de combat...... JUJU  :p\\


		Si c'est un noeud terminal, une ChoiceBox est juste rajouté afin de savoir si c'est un noeud gagnant ou perdant. Lors de la validation, un BookNodeTerminal est créer avec comme valeur un BookNodeStatus défini en fonction de la ChoiceBox.\\

		Dès qu'un noeud est validé, un NodeFx est créer. Il contient alors le noeud créer et affiche un Rectangle là où la souris a été cliqué dans le graphPane, à la création du noeud. La couleur du carré change en fonction du type de noeud. Un RectangleFX est donc mis en place permettant d'envoyé une notification si jamais la souris passe dessus (opacité du Rectangle qui change), réalise un clique maintenu (déplacement du rectangle en fonction de la souris), réalise un clique simple. Pour ce dernier, cela envoie une notification à l'observable permettant de prévenir l'observeur qu'on a cliquer dessus. Cela sert si un double clique est produit (permet de procédé à une modification si le mode activé est "SELECT", a une suppression si c'est le mode DELETE qui est activé ou encore a une création de lien entre le prélude et le noeud sélectionné si le mode FIRST_NODE est activé) ou si un deux cliques espacé sont réalisé (permet de procédé à la création d'un lien si le mode "ADD_NODE_LINK" est sélectionné).\\

		Une fois un ou plusieurs noeud créer, un lien peut être donc être effectué. Le lien peut être fait sur le même noeud, si une boucle est voulu. Mais il peut aussi être réalisé entre deux noeuds. Pour ce faire, le mode "ADD_NODE_LINK" doit être activé. Le premier clique permet de sauvegarder le noeud de départ. Le deuxième permet d'avoir le noeud de destination. Une fois les deux cliques détecté, une fenêtre de dialog apparait. Il y a trois affichage différent.\\
		L'affiche ce fait en fonction du premier noeud. Tout d'abord, l'affichage commun entre ces noeuds est un simple texte, deux TextField ainsi qu'un CheckBox. Les TextField permettent d'ajouter un gain ou une perte de vie/d'argent. La CheckBox, quant a elle, permet d'aller vers ce choix obligatoirement ou non.\\
		Si le premier noeud est un noeud de combat, une ChoiceBox apparait alors permettant de choisir si c'est un lien gagnant, perdant ou un lien d'évasion. Cette liste de choix est créer si le permier noeud à encore des choix de libre. S'il n'a plus de choix libre, une boite d'alerte s'affiche en prévenant qu'il n'y a plus aucun lien disponible.\\

		Si le permier noeud est de type aléatoire, le joueur doit alors renseigner la chance sur chaque lien.\\

		Le lien est repésenté par une ligne entre le noeuds de départ et d'arrivé. Pour les différencier, un cercle est créer et est affiché au noeud de destination. Tout cela est géré par NodeLinkFx qui a aussi un observeur permettant d'envoyé une notification si jamais la ligne ou le cercle enregistre un événement "clique". Cela permet de savoir si le lien a été double cliqué pour réaliser une modification.\\

		Le prélude est déjà situé en haut à gauche du GraphPane. Ce prélude, comme les noeuds, est représenté par un carré. Ce carré est lié à un PréludeFx contenant des Tab. La première Tab contient le texte du prélude. La deuxième contient la création du personnage. Et la troisième la création du personnage principal.\\
		La deuxième partie du TabPane contient un Accordion.  Cette accordion est affiché grâce au bouton "Ajouter". Chaque accordion est composé d'un CharacterCreationComponent comprenant un Texte, une CheckBox (Texte, Item), et d'un bouton "Supprimer cette partie". La checkBox n'affiche que "Texte" si aucun item n'est disponible. Mais si un item est créé, l'utilisateur peu alors ajouter un ItemListComponent en sélectionnant "Item" dans la CheckBox. Cette classe est affiché sur tout les noeuds, permettant d'ajouter des items ainsi que la quantité disponible de chaque item, qui ici, sera disponible en début de partie. L'item ajouté peut être supprimé grâce à un clic droit de la souris.
		La troisième partie du TabPane permet quant a elle, de créer le personnage principal grâce au CharacterComponent. Cette classe permet de créer un personnage. Mais si la valeur boolean de mainCharacter est true, les TextField pour le nombre d'item maximum ainsi que la quantité d'argent est ajouté.
		Après validation, le textPrelude (texte du prélude), mainCharacter (personnage principal) et le characterCreations (item et texte à afficher après le prélude) est créer.

		\subsection{RightPane}

		Ce panel contient tout ce qui représente les statistiques. Dès qu'un noeud est ajouté ou supprimé, une méthode de la class \textit{RightPane} est utilisé afin de mettre à jour le nombre de noeuds existant. Pour la difficulté du livre, elle est mise à jours dès que \textbf{Estimé la difficulté du livre} est sélectionné.\\
		Si un nouveau livre est chargé, tout ces statistiques ce remmettent à zéro.



	\section{Rendre le livre jouable et estimer sa difficulté}
		\subsection{Jeu}
		Une classe à été créer se nommant \textbf{Jeu}, permettant de gérer les méthodes de jeu communes entre le \textit{Player} et les \textit{Fourmis}.\\
		Un construteur est d'abord appelé, à partir de la MainWindows, afin d'envoyer le livre contenant toutes les informations. Puis, en fonction du mode sélectionner ("Générer la difficulté" ou "jouer"), la méthode correspondante au player est appelé.\\
		Une fois dans la méthode choisis, le livre est alors copié afin de ne pas le modifier par erreur pendant le jeu. Un BookState, qui correspond à la sauvegarde de la partie, est alors créé. Si le prélude contient un personnage principal, alors celui ci est enregistré dans la sauvegarde de la partie (le BookState). Dans le cas contraire, un autre personnage principal est créer afin de pouvoir jouer au jeu.\\
		Enfin, si des compétences et/ou des items sont disponible au début de la partie, la méthode de création de joueur est appelé en focntion du player actuel. Comme cela, le player choisit parmi une liste de skill et d'items disponible en début de partie afin de les avoir pour commencer le jeu.\\
		Une fois le BookState créer, le personnage principal initialisé et la copie du livre enregistré, le premier noeud est donc chargé. Une méthode est appelé en fonction de son type de noeud. La méthode correspondante au type de noeud s'exécute et renvoie le noeud de "destination", en fonction du choix du player et/ou de la mort du player. Durant l'exécution des différentes méthodes et en fonction du player, d'autre méthode externe sont appelé nottament dans la classe Fourmis ou Player.\\
		Pour chaque type de noeud, sauf pour un noeud terminal, une méthode commune est appelé afin de savoir si le noeud pris en charge fait gagné/perdre de la vie puis regarde si le player est toujours en vie. Si ce dernier n'est plus en vie, un noeud terminal est alors renvoyé en noeud de destination. S'il est encore en vie et que le noeud propose des items, ils sont proposés au player en appelant la méthode correspondante entre Fourmis ou Player. A chaque detination choisi, une autre méthode est appelé afin de regarder si le lien entre le noeud de départ et de destination fait perdre ou gagner de la vie et/ou de l'argent.\\

		\underline{Si un noeud est de type basic}, il est alors pris en charge dans la méthode \textit{execNodeWithChoices}. Cette dernière renvoi un noeud terminal si aucun choix n'est valide, ce qu'il veut dire, si le player n'a aucun choix ou s'il ne possède pas les items/skill pour aller vers ce choix.\\
		Si le player est encore en vie et si il peut au moins choisir une destination, un choix est demandé parmis toutes les destinations faisant un appel à la méthode en fonction du player. Si le player a les prérequis pour aller vers cette destination, alors le noeud choisi est renvoyé en noeud de destination. Sinon, le player doit faire un autre choix.\\

		\underline{Si c'est un noeud de type combat}, une vérification est réalisé afin savoir si le noeud contient des ennemis. S'il n'y a pas d'ennemis, le noeud en cas de victoire est envoyé en noeud de destination. Sinon, une liste d'ennemis est créer afin de ne pas modifié la vie des ennemis. Car ces derniers ne sont pas lié au noeud, mais c'est l'ID de l'ennemis qui est lié au noeud permettant de les appelés plusieurs fois dans plusieurs ou dans le même noeud.\\
		Le combat commence alors. Le choix est défini par la méthode du player correspondant. Trois choix sont possible:\\
		\begin{itemize}
			\item \underline{Attaque :} un autre choix est demandé permettant de sélectionner l'ennemi à attaquer parmi la liste des ennemis encore en vie. Une fois l'ennemi sélectionné, une méthode attaque est appelé apellant elle même une autre méthode commune entre l'attaque du player et l'attaque d'ennemi. Nommé \textif{getDamageAmount}, elle permet de savoir le nombre de dommage réalisé en fonction des point d'attaque de l'attaquant, de son double dommage décider en random si ce boolean est défini en true, d'un coup critique décider aussi en random, de l'arme de l'attaquant et de l'item de défense de l'attaqué. L'attaquant et l'attaqué est défini en fonction de la permière méthode qui l'appel. Ici c'est la méthode d'attaque du player.\\
			Une fois l'attaque effectué, si l'ennemi attaqué est mort, il est supprimer de la liste des ennemis.
			\item \underline{Inventaire :} si ce choix est fait par le player, la méthode appelé permettant d'utiliser son inventaire est elle même gérer dans la classe Player ou la classe Fourmis. Elle permet alors de choisir une potion, une arme et ou un item de défense.
			\item \underline{Evasion :} si le tour avant évasion est inférieur ou égal à 0 et si un noeud de d'évasion existe, le player peut alors s'enfuir. Sinon cela lui passe son tour.
		\end{itemize}
		\\
		Une fois le tour du joueur fini, vient le tour de l'ennemi. Il appel une méthode envoyant la liste d'ennemis restant. Cette méthode appel \textif{getDamageAmount} permettant aux ennemis d'attaquer un par un.\\

		La fin de combat est déterminé si la liste d'ennemis est vide ou si le player n'est plus en vie. Le noeud de destination est alors défini en fonction du résultat en fin de combat.\\


		\underline{Si le noeud est de type aléatoire}, la méthode commune est appelé afin de savoir si le player est encore en vie. Puis une autre méthode est appelé afin de déterminer le noeud de destination en fonction des chances attribué à chacun de ces choix.\\

		\underline{Si le noeud est de type terminal}, la partie est alors terminé et renvoie un bolean sur l'état de la fin de partie.\\

	\subsection{Interface Player / Foumis}
		Une interface \textbf{InterfacePlayerFoumis} à été créer permettant une mise en commun des codes Player et Fourmis. Ces méthodes permettent de faire un choix, prendre les items disponibles, créer un personage lambda, aller dans l'inventaire, choisir son ennemis ou encore combatre. Elles sont appelé au même moment. La méthode sera alors exécuté différément en fonction du player.\\

		\textit{execPlayerCreation} permet de choisir les skill et les items disponible au début de la partie. Ces derniers sont défini lors de la création du prélude. Pour l'ajout des items, la méthode \textit{prendItems} est appelé.\\

		\textit{combatChoice}, prend en paramètre le noeud de Combat et le nombre de tour avant l'évasion ainsi que le BookState, permet de faire un choix lors du tour du player dans un combat. On peut alors choisir d'attaquer, d'aller dans l'inventaire ou alors de s'évader. Si on choisi l'inventaire, on va alors dans une autre méthode appelé useInventaire() qui prend le BookState en paramètre. On peut alors utiliser une potion, prendre un objet de défense ou alors une arme. Si l'on choisis un autre choix, cette objet n'est pas utilisable lors d'un combat (comme par exemple de l'argent). Une fois l'objet pris, on retourne dans les choix du combat. On peut alors, soit retourner dans l'inventaire pour prendre un autre objet, soit attaquer ou s'évader.\\

		\textit{chooseEnnemi} permet de choisir l'ennemi à attaquer parmi la liste de tout les ennemis encore en vie.\\

		\textit{prendItems} permet de prendre un item parmi la liste d'items disponible. Cette liste est pris en paramètre ainsi que la sauvegarde de la partie et le nombre d'item maximum pouvant être pris.\\

		\textit{makeAChoice} permet de faire un choix en fonction des différentes destinations proposé par le noeud.\\

		\textit{useIventaire} permet d'utiliser son inventaire lors d'un noeud de combat. La mise à jour d'un port d'item de défense ou d'arme est alors mis à jour. Si un item de soin est choisi, les points de vie du joueur sont alors actualisé.\\

	\subsection{Player}
		La classe \textbf{Player} permet de jouer au jeu en tant que joueur. Elle permet de faire des choix grâce aux Scanner. Des messages sont aussi affiché afin de guider le joueur dans ses choix.\\
		Notament la méthode \textit{choixYesNo} qui permet de choisir oui ou non et de renvoyer le boolean true ou false. Cette méthode permet, par exemple, de savoir si le player veut supprimer, prendre un item ou un skill.\\
		Pour la méthode commune \textit{prendItems}, cette dernière fait appel à d'autre méthode dans la classe Player. Comme itemAdd() permettant de choisir l'item à ajouter dans l'inventaire. Ou encore itemPlein() qui demande au joueur s'il veut supprimer un item. Si le joueur répond oui, la méthode itemSupp() est appelé afin de choisir l'item à supprimer et à mettre à jour l'inventaire.\\
		Pour la méthode \textit{execPlayerCreation} au moment de l'ajout des skill, le joueur doit confimer ou non s'il veut un skill. Si oui, la méthode skillAdd est appelé jusqu'à ce que le maximum d'item à été pris ou qu'ilne reste plus de skill à prendre.\\

	\subsection{Fourmis}
		La classe \textbf{Fourmis} permet de jouer en tant que joueur fictif. Elle effectue des choix random en fonction des différentes méthodes de l'interface.
		Comme par exemple, pour prendre des items ou des skill, la fourmi en prend autant que possible et en supprime obligatoirement en aléatoire si il n'a plus de place dans l'inventaire. Nous avons décider de réalisé cette méthode comme cela afin de pouvoir aller dans le maximum de noeud s'ilsont des prérequis ou alors avoir le maximum d'items pour les combats.\\
		Pour la méthode \textit{chooseEnnemi}, la fourmi envoyé prend obligatoirement le premier ennemi permettant de tué le maximum d'ennemis en attaquant toujours le même ennemis.\\
		Et enfin, la méthode \textit{combatChoice} permet de choisir entre ATTAQUE, EVASION, INVENTAIRE. Nous avons choisir de faire un random sur les trois choix et non pas sur deux choix même si le tour d'évasion n'est pas disponible afin de passer le tour, comme le joueur, afn d'avoir la même chance lors des combats.\\
