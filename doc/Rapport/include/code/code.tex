\chapter{L'application et le code}



	\section{Présentation des packages}



	\section{Diagramme de classe}



	\section{Aspects techniques}

		\subsection{Jeu,Player,Foumis}
			\headline
			\underline{Jeu}
			Une classe à été créer se nommant \textbf{Jeu}, permettant de gérer les méthodes de jeu communes entre le \textit{Player} et les \textit{Fourmis}.\\
			Un construteur est d'abord appelé afin d'avoir le livre commun à toute les classes. Puis, celui le mode sélectionner ("Générer la difficulté"" ou "jouer"), on fait appels à la méthode correspondante au player. Une fois quele mode à été cliqué, le livre est alors copié afin de ne pas le modifier dans la classe au cas où. Un BookState, correspondant à la sauvegarde de la partie, est alors créer à partir du BookCharacter généré par le prélude. C'est donc le personnage principal. Si aucun personnages n'est créer, alors un personnage lambda va étre créer afin de pouvoir jouer au jeu.\\
			Une fois le BookState créer et la copie du livre enregistrer, on prend le premier paragraphe et on regarde à quel "noeud" il appartient. Une méthode sera ainsi appeler en fontion du type de noeuds qui prend en charge.\\
			La méthode correspondante au type de noeud s'exécute et renvoie le noeud de "destination", en fonction du choix du player, ou de la mort du player. En effet, ces "noeuds" peuvent faire venir la mort du player en enlevant de la vie par exemple, ou que ce player tombe dans une embuscade... Ces noeuds offre beaucoup de possibilité.\\

			Durant l'exécution de la méthode, et en fonction du player, d'autre méthode externe sont appeler, nottament dans la classe Fourmis ou Player.

		\underline{Interface Player / Foumis}
			Une interface \textbf{InterfacePlayerFoumis} à été créer permettant une mise en commun des codes Player et Fourmis. Ces classes permettent de faire un choix, prendre les items disponibles, créer un personage lambda, aller dans l'inventaire, choisir son ennemis ou encore combatre.\\
			Elles permettrent de d'appeler la même méthode (que cela soit fourmis ou player) au même moment. La méthode sera alors exécuté différément en fonction du player. Cela permet donc une harmonie du code

		\underline{Player}
			La classe \textbf{Player} permet de jouer au jeu en tant que joueur. Elle permet de faire des choix grâce aux Scanner.\\ Cette classe a des méthodes de l'interface, notamment celle de combatChoice qui prend en paramètre le noeud de Combat, le nombre de tour avant l'évasion ainsi que le BookState. Cette méthode permet de choisir nos choix lors de notre tour dans le combat. On peut alors choisir d'attaquer, d'aller dans notre inventaire ou alors de s'évader.\\
			Si on choisi l'inventaire, on va alors dans une autre méthode appelé useInventaire() qui prendre le BookState en parametre. On peut alors utiliser une potion, prendre un objet de défense ou alors une arme. Si l'on choisis un autre choix, cette objet n'est pas utilisable lors d'un combat (comme par exemple de l'argent). Une fois l'objet pris, on retourne dans les choix du combat. On peut alors, soit retourner dans l'inventaire pour prendre un autre objet, soit attaquer ou s'évader.\\
			Si le choix évasion est choisi, un message apparait si le nombre de tour avant l'évasion n'est pas à zero. Si il n'est pas à zéro, un message apparait et il doit refaire un autre choix. Sinon, il va alors dans le noeud de destination qui a été prévu pour l'évasion.\\
			Si le choix attaque est choisi...


		\underline{Fourmis}

		\subsection{Edition du livre}
