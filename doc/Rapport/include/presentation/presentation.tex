\chapter{Présentation du projet}

	\section{Présentation de l'application}
        Magik Book est un éditeur de livre. Il permet donc de créer un livre à choix, avec des conditions pour certains choix.\\
        On peut donc créer des paragraphes, appeler des "noeuds", de différents types: choix simple, choix de chances (random), combat, terminaux (victoire ou défaite). Cette aplication comprend aussi la création d'un pélude ainsi que des personnages et d'items.\\
        Une fois le livre créer, nous pouvons alors regarder sa difficulté en choisissant dans la bar des menu en haut. Cette difficulté est affiché alors dans le panel des stats. Une bouton pour jouer est également disponible afin de pouvoir profité pleinement de l'hitoire créer.\\
        Nous pouvons enregistrer notre livre en format json et le réouvrir afin de pouvoir continuer l'édition de ce dernier.


	\section{Choix des technologies}

		\subsection{Git}


		\subsection{Gradle}



		\subsection{JavaFx}



	\section{Organisation du projet}



		\subsection{Dossier}



		\subsection{GitHub et Forge}



		\subsection{Trello}



		\subsection{Discord}
