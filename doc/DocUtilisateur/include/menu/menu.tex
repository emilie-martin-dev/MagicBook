\chapter{Menu}

	\section{Fichier}
		\begin{itemize}
			\item \textbf{Nouveau} Création d'un nouveau livre.
			\item \textbf{Ouvrir} Ouverture d'un fichier en format Json.
			\item \textbf{Enregistrer} Enregistrement du fichier en format Json.
			\item \textbf{Enregistrer-Sous} Enregistrement-sous du fichier en format Json.
		\end{itemize}

	\section{Livre}
		\begin{itemize}
			\item \textbf{Jouer} Permet de jouer au livre.
			\item \textbf{Estimer la difficulté} Permet d'estimer la difficulté et de l'afficher dans le panel des stats.
			\item \textbf{Générer le livre en format txt} Génère le livre en format texte. Permet à l'utilisateur d'avoir un livre lisible.
		\end{itemize}

	\section{Affichage}

		\subsection{Agrandir la zone d'édition}
			\begin{itemize}
				\item \textbf{Mode,Items,Personnages} Affiche/Cache le panel de gauche
				\item \textbf{Statistiques} Affiche/Cache le panel de droite
			\end{itemize}

		\subsection{Zoom/dezoom}
			Un zoom/dézoome peut est possible grâce à la molette de la souris, permettant ainsi d'agrandir la visibilité sur la zone d'édition.
